% SatNOGS COMMS System Design Document - User Manual
% Copyright (C) 2020, 2022 Libre Space Foundation
%
% This work is licensed under a
% Creative Commons Attribution-ShareAlike 4.0 International License.
%
% You should have received a copy of the license along with this
% work.  If not, see <http://creativecommons.org/licenses/by-sa/4.0/>.

% Linter settings
% chktex-file -2

\documentclass[english,title,a4paper]{report}
\usepackage[utf8]{inputenc}
\usepackage[T1]{fontenc}
\usepackage[english]{babel}
\usepackage{float}
\usepackage[hidelinks]{hyperref}
\usepackage[toc,acronyms,section,nopostdot,nonumberlist,section=section,numberedsection]{glossaries}
\usepackage[toc]{appendix}
\usepackage{graphicx}
\usepackage{lastpage}
\usepackage{titlesec}
\usepackage{fancyhdr}
\usepackage{lipsum}
\usepackage{listings}
\usepackage{booktabs}
\usepackage[maxfloats=500]{morefloats}
\usepackage{lscape}
\usepackage{makecell}
\usepackage[table]{xcolor}
\usepackage{fixme}
\usepackage{pdfpages}
\usepackage{xcolor}

\definecolor{codegreen}{rgb}{0,0.6,0}
\definecolor{codegray}{rgb}{0.5,0.5,0.5}
\definecolor{codepurple}{rgb}{0.58,0,0.82}
\definecolor{backcolour}{rgb}{0.95,0.95,0.92}

\newcommand\TBD[1]{\textcolor{red}{#1}}

\lstdefinestyle{mystyle}{
    backgroundcolor=\color{backcolour},
    commentstyle=\color{codegreen},
    keywordstyle=\color{magenta},
    numberstyle=\tiny\color{codegray},
    stringstyle=\color{codepurple},
    basicstyle=\ttfamily\footnotesize,
    breakatwhitespace=false,
    breaklines=true,
    captionpos=b,
    keepspaces=true,
    numbers=left,
    numbersep=5pt,
    showspaces=false,
    showstringspaces=false,
    showtabs=false,
    tabsize=2
}

\lstset{style=mystyle}

\fancypagestyle{plain}{
  \fancyhf{}
  \fancyhead[L]{SatNOGS COMMS}
  \fancyhead[R]{User Manual}
  \fancyfoot[L]{Version \version\textit{DRAFT}}
  \fancyfoot[C]{\thepage{} of~\pageref{LastPage}}
  \fancyfoot[R]{\copyright{} Libre Space Foundation}
  \renewcommand{\headrulewidth}{0.4pt}
  \renewcommand{\footrulewidth}{0.4pt}
}
\pagestyle{plain}
\setcounter{secnumdepth}{3}
\setcounter{tocdepth}{3}

\input{include/version}

\makeglossaries{}

%% Acronyms
\input{glossaries/acronyms}

\title{
  User Manual \\
  \large{SatNOGS COMMS} \\
}
\author{Libre Space Foundation}
\date{Version \version\\\textit{DRAFT}}

\titleformat{\chapter}[hang]
{\normalfont\bfseries\Huge}{\thechapter.}{10pt}{}

\begin{document}

\hypersetup{pageanchor=false}
\maketitle
\hypersetup{pageanchor=true}
\tableofcontents
\newpage

\chapter{Changelog}

\begin{tabular}{llp{6.5cm}}
    \toprule
    Version & Date          & Changes                                     \\ \midrule
    1.0     & Dec 18, 2023  & No Changes                                  \\ \midrule
    1.1     & Dec 29, 2023  & No Changes                                  \\ \midrule
    1.2     & Mar 19, 2024  & No Changes                                  \\ \midrule
    1.3     & Apr 30, 2024  & No Changes                                  \\ \bottomrule
\end{tabular}

\chapter{Preface}\label{preface}

SatNOGS COMMS is an open hardware and open source software project.
This allows the end user to customize all aspects of the \acrshort{mcu} and \acrshort{fpga} Software to meet their needs.

Further information regarding software can be found in the software repositories under
\url{https://gitlab.com/librespacefoundation/satnogs-comms/}

\chapter{Unpacking and handling}\label{unpacking-and-handling}

SatNOGS COMMS is and \acrshort{esd} sensitive device.
Standard \acrshort{esd} handling methods should be followed when handling the \acrshort{pcb}\@.

Always use an \acrshort{esd} mat and \acrshort{esd} wrist strap.
It is advised to wear gloves when handling the PCB and shield to avoid contamination.

Should cleaning of any parts is required, use \acrshort{esd} safe cleaning methods and non-reactive solvents.

\chapter{Hardware setup}\label{hardware-setup}

\section{RF Connection}\label{rf-connection}

SatNOGS COMMS should always be powered on with an antenna or RF load connected to its \acrshort{rf} ports.
Failing to do so may result in catastrophic failure of the system.

A 50 Ohm antenna or 50 Ohm \acrshort{rf} load capable of sinking 33dBm of power must be present to each \acrshort{rf} connector at all times.

\section{Power supply}\label{power-supply}

SatNOGS COMMS can operate over a voltage range of 6 to 14V DC \@.
This can be provided either through a power supply or directly from the satellite battery.

A power supply should be connected to VIN and GND pins as described in ICD \@.
Adjust the current limit so that a maximum power of \TBD{xxW} is available.

\chapter{Communication interfaces}\label{communication-interfaces}

\section{Programming interface}\label{programming-interface}

\acrshort{mcu} can be programmed and debugged via the provided SWD/JTAG interface available in connector Jxxx

\section{External connectivity}\label{external-connectivity}

\subsection{CAN Bus}\label{can-bus}

There are two CAN Bus channels that can be configured either as standard CAN or CAN FD for speeds up to 5Mbps.
COMMS uses CAN Bus to communicate with other subsystems and relay data or receive commands.

\subsection{I2C}\label{i2c}

I2C is provided for communication with I2C capable devices like antenna deployers, robes etc.

\subsection{SPI}\label{spi}

COMMS provides an SPI slave connection for up to 8 Mbit data transfer speeds.
Slave communication is coordinated via CAN Bus instead of a dedicated slave select line.
There are 2 CAN Bus commands for SPI control:

\begin{enumerate}
\def\labelenumi{\arabic{enumi}.}
  \item SPI Select
  \item SPI Release
\end{enumerate}

Proper SPI communication sequence is:

\begin{enumerate}
\def\labelenumi{\arabic{enumi}.}
  \item Issue ``SPI Select'' command
  \item Wait for acknowledge
  \item Preform SPI communication
  \item Issue ``SPI Release'' command
  \item Wait for acknowledge
\end{enumerate}

\subsection{Antenna}\label{antenna}

There are two antenna control lines and two antenna feedback lines available for managing antenna deployment.

These are available on the dedicated antenna control connector as well as on the PC104 connector.

Available commands are:

\begin{itemize}
  \item Antenna 1 control On/Off
  \item Antenna 2 control On/Off
  \item Antenna Feedback Request
\end{itemize}

Antenna Control Feedback is broadcast on CAN Bus when any of the two feedback lines changes state or an Antenna Feedback Request command is received.

\section{RF Interfaces}\label{rf-interfaces}

\chapter{Data Link Layer (Layer 2)}\label{data-link-layer-layer-2}

Data link layer utilizes CCSDS Space Packet implemented by Open Space Data Link Protocol.
See \url{https://gitlab.com/librespacefoundation/osdlp} for further information.

\chapter{Physical Layer (Layer 1)}\label{physical-layer-layer-1}

BPSK, FSK, MSK IEE802.15.4, CCSDS

\section{Basic RF Mode}\label{basic-rf-mode}

\subsection{Modulation}\label{modulation}

\subsection{Frequency}\label{frequency}

TX and RX at 390 MHz to 500 MHz

TX at 2200 MHz to 2290 MHz

RX at 2025 MHz to 2110 MHz

\subsection{Bandwidth}\label{bandwidth}

\section{Extended RF Mode}\label{extended-rf-mode}

\subsection{Modulation}\label{modulation-1}

\subsection{Frequency}\label{frequency-1}

\subsection{Bandwidth}\label{bandwidth-1}

\chapter{Command and status messaging}\label{command-and-status-messaging}

\section{Commands}\label{commands}

\subsection{Antenna control}\label{antenna-control}

\subsection{Antenna feedback request}\label{antenna-feedback-request}

\subsection{SPI Select}\label{spi-select}

\subsection{SPI Release}\label{spi-release}

\section{Messages}\label{messages}

\subsection{Antenna Control Feedback}\label{antenna-control-feedback}

\chapter{Self-protection functions}\label{self-protection-functions}

\section{Temperature supervisor}\label{temperature-supervisor}

There are several temperature sensor monitoring key components of COMMS \@.
If an over-temperature event occurs, COMMS will disable the affected part of the circuit until temperature is within the operating envelope.

\section{Power supervisor}\label{power-supervisor}

COMMS power lines are monitored for over-current conditions.
If an over-current event occurs, the affected part of the circuit can be disabled allowing the rest of COMMS systems to operate.

COMMS power circuits incorporate continuous voltage monitoring of all onboard power supplies.

There is a mechanism for mitigating \acrshort{seu} of MOSFETS in the DC-DC converters.
Should a \acrshort{seu} occur, COMMS will power down and will need to be power cycled by the EPS in order to recover from the \acrshort{seu} \@.

\chapter{Telemetry and Telecommand}\label{telemetry-and-telecommand}

The telemetry and telecommand interface of the SatNOGS COMMS follows the \href{https://public.ccsds.org/Pubs/232x1b2e2c1.pdf}{CCSDS 232.1-B-2} utilizing the functionality and capabilities of the CCSDS Space Data Link.

The interface software between the ground station and the SatNOGS COMMS is the \href{https://gitlab.com/librespacefoundation/osdlp-operator}{OSDLP-Operator}.
This software, exploits the capabilities of the Space Link layer and implements all the necessary validity checks, re-transmissions, state transitions, priorities, virtual channel and session management, totally abstracting any implementation details from the ground station operator.

The CCSDS Communications Operation Procedure specifies the Frame Operation Procedure (FOP) that executes within the sending entity, and a Frame Acceptance and Reporting Mechanism (FARM) that executes within the receiving entity.

The OSDLP-Operator can support an arbitrary number of FOP and FARM configurations, using a descriptive text-based configuration language.
Listings 1 and 2 provide example configurations for FARM and FOP respectively.
The configurations can be altered without requiring any modification on the OSDLP-Operator software.
This provides great flexibility on the ground station control software and allows for rapid development and prototyping, as well as fast reaction times during the in-orbit period.

\begin{lstlisting}[language=C, caption = FARM example configuration]
    // Declare which instance to run. For ground station use "fop".
    // For receiving side (e.g. spacecraft) use "farm"
    instance = "farm"

    // Mission Parameters
    mission:{
      tc_sent_queue_max_cap = 200  // Maximum capacity of TC sent queue
      tc_tx_queue_max_cap = 200    // Maximum capacity of TC TX queue
      tc_rx_queue_max_cap = 200    // Maximum capacity of TX RX queue
      scid = 20                    // The spacecraft ID
      out_port = 6667
      in_port = 6666
    }

    // Define the TC parameters
    tc:{
    // Define one such struct per virtual channel
      vc1:{
      // The following parameters apply to the whole VC
         vcid = 1                   // The VC ID
         crc = 1                    // CRC flag. 1 for on, 0 for off
         segmentation = 0           // Defines whether to use segmentation

         // Defines the available MAPs. Each MAP is a channel contained into
         // the virtual channel. The following params apply to each MAP
         map:{
           map1:{
             mapid = 1             // The MAP ID
             bypass = 1            // Bypass flag. 0 for TYPE_A ->use the
                                   // sequence checking mechanism. 1 for
                                   // TYPE_B -> bypass sequence checking
                                   // mechanism
             ctrl_cmd = 1          // Control command flag. 0 for mission
                                   //specific commands/data, 1 for FARM
                                   // control information
             data = ""             // For future use
           }
         }
         // In case of farm instance, define this element
         farm:{
           win_width = 5           //Sliding window width
         }
      }

      vc2:{
         vcid = 2
         crc = 1
         segmentation = 0
         map:{
           map1:{
             mapid = 1
             bypass = 1
             ctrl_cmd = 1
             data = ""
           }
         }
         farm:{
           win_width = 5
         }
      }
    }

    // Define the TM parameters
    tm:{
    // One such element per virtual channel
      vc1:{
         vcid = 1             // The virtual channel ID
         crc = 1              // CRC flag. 1 for on, 0 for off
         ocf_flag = 1         // OCF flag. 1 for on, 0 for off
         sec_hdr_on = 0       // Secondary header flag. 1 for on, 0 for off
         sync_flag = 0        // Sync flag
         stuff_state = 1      // Stuff packets state. 0 for ON 1 for OFF
      }
    }
\end{lstlisting}

\begin{lstlisting}[language=C, caption = FOP example configuration]
    instance = "fop"

    mission:{
      tc_sent_queue_max_cap = 200
      tc_tx_queue_max_cap = 200
      tc_rx_queue_max_cap = 200
      scid = 1
      out_port = 16881
      in_port = 16880
      stdout_port = 16882        // UDP port for printing log messages
    }

    tc:{
      vc0:{
         vcid = 0
         crc = 1
         segmentation = 1
         vc_name = "Management"
         map:{
           map1:{
             mapid = 1
             bypass = 0
             ctrl_cmd = 0
             data = "xA0x0Cx20x0CxA0x0Cx20x0CxA0x0Cx20x0CxA0x0Cx20x0CxA0x0Cx20x0CxA0x0Cx20x0C"
             name = "Change periodic telemetry attributes"
           }
           map2:{
             mapid = 2
             bypass = 0
             ctrl_cmd = 0
             data = "xff"
             name = "Request periodic telemetry attributes"
           }
           map3:{
             mapid = 3
             bypass = 0
             ctrl_cmd = 0
             data = ""
             name = "Kill switch"
           }
           map4:{
             mapid = 4
             bypass = 0
             ctrl_cmd = 0
             data = "xff"
             name = "Deploy antenna"
           }
         }
         //In case of fop instance, define this element
         fop:{
           t1_init = 5                // The initial timer value in sec
           transmission_limit = 20    // Transmission limit per packet
           timeout_type = 0           // Timeout type
           win_width = 5              // Sliding window width
         }
      }
        vc2:{
         vcid = 2
         crc = 1
         segmentation = 1
         vc_name = "Request TM"
         map:{
           map1:{
             mapid = 1
             bypass = 0
             ctrl_cmd = 0
             data = "xa4"
             name = "Request TM"
           }
         }
         fop:{
           t1_init = 5
           transmission_limit = 30
           timeout_type = 0
           win_width = 20
         }
      }
       vc3:{
         vcid = 3
         crc = 1
         segmentation = 1
         vc_name = "Experiment"
         map:{
           map1:{
             mapid = 1
             bypass = 0
             ctrl_cmd = 0
             data = "x01xFEx02x01x04x00x0A"
             name = "Execute experiment"
           }
         }
         fop:{
           t1_init = 5
           transmission_limit = 30
           timeout_type = 0
           win_width = 5
         }
      }
    }
    tm:{
      vc0:{
         vcid = 0
         crc = 1
         ocf_flag = 1
         sec_hdr_on = 0
         sync_flag = 0
         stuff_state = 1
         vc_name = "Management"
      }
      vc1:{
         vcid = 1
         crc = 1
         ocf_flag = 0
         sec_hdr_on = 0
         sync_flag = 0
         stuff_state = 1
         vc_name = "Regular TM"
      }
        vc2:{
         vcid = 2
         crc = 1
         ocf_flag = 1
         sec_hdr_on = 0
         sync_flag = 0
         stuff_state = 1
         vc_name = "Request TM"
      }
        vc3:{
         vcid = 3
         crc = 1
         ocf_flag = 1
         sec_hdr_on = 0
         sync_flag = 0
         stuff_state = 1
         vc_name = "Experiment"
      }
    }
\end{lstlisting}

\section{Ground control setup}\label{ground-control-setup}

The provided ground control software allows telemetry reception and telecommand of SatNOGS COMMS using a \acrshort{sdr} transceiver.

\section{Telemetry format}\label{telemetry-format}

\section{Telecommand format}\label{telecommand-format}

\section{SatNOGS integration}\label{satnogs-integration}

\chapter{Diagnostics}\label{diagnostics}

\end{document}
